\documentclass[acmsmall,nonacm,review,anonymous]{acmart}

% template:

%\setcopyright{acmcopyright}
\setcopyright{none}
\copyrightyear{2018}
\acmYear{2018}
\acmDOI{XXXXXXX.XXXXXXX}
\acmJournal{JACM}
\acmVolume{37}
\acmNumber{4}
\acmArticle{111}
\acmMonth{8}



\usepackage{amsmath,amsfonts}
\usepackage{algorithmic}
\usepackage{graphicx}
\usepackage{textcomp}
\usepackage{xcolor}
\usepackage{hyperref}
\usepackage{import} % subimport
\usepackage[nameinlink]{cleveref}
\usepackage{ifthen} % for todos
\usepackage{xspace}
\usepackage{framed}
\usepackage{subcaption} % subfigure
\usepackage{listings}
\usepackage{soul} % for hl
\usepackage{wasysym}
\usepackage{booktabs}
\usepackage{tabularx}
\usepackage{multirow}
\usepackage{pifont}
\usepackage{stmaryrd} % llbracket and rrbracket
\usepackage[ruled,vlined]{algorithm2e}
\usepackage{minted}

\newtheorem{definition}{Definition}
\newtheorem{proposition}{Proposition}

\def\algorithmautorefname{Algorithm}

\newcolumntype{Y}{>{\centering\arraybackslash}X} % https://tex.stackexchange.com/questions/60601/evenly-distributing-column-widths

\lstset{
xleftmargin=0in,
basicstyle=\ttfamily\lst@ifdisplaystyle\footnotesize\fi,
breaklines=true
}
\definecolor{myred}{HTML}{aa0000}
\definecolor{mygreen}{HTML}{008000}
\lstdefinelanguage{diff}{
  morecomment=[f][\color{mygreen}]{+\ },
  morecomment=[f][\color{myred}]{-\ },
}
\definecolor{gray}{RGB}{215,215,215}
\sethlcolor{gray}

\newcommand{\projName}{\textsc{TermFix}\xspace}
\newcommand{\noprior}{\textsc{NoP}\xspace}
\newcommand{\dta}{\textsc{LinearRankTerm}\xspace}
\newcommand{\rfs}{\textsc{RankFinder}\xspace}
\newcommand{\dtaRepair}{\textsc{LinearRankTermRepair}\xspace}
\newcommand{\rfsRepair}{\textsc{RankFinderRepair}\xspace}
\newcommand{\reproductionURL}{XXX} %{\url{https://doi.org/XXX/zenodo.XXX}}

% TODOs
\newboolean{showcomments}
\setboolean{showcomments}{true} % comment this line to deactivate comments
\ifthenelse{\boolean{showcomments}}{
  \newcommand{\nbc}[3]{
    {\textcolor{#3}{\small{\bfseries{#1:\ }}\textit{#2}}}}
}{
  \newcommand{\nbc}[3]{}
}

\newcommand{\sergey}[1]{\nbc{@Sergey}{#1}{red}\xspace}
\newcommand{\peter}[1]{\nbc{@Peter}{#1}{blue}\xspace}
\definecolor{todogreen}{rgb}{0.15, 0.65, 0.25}
\newcommand{\robbie}[1]{\nbc{@Robbie}{#1}{green}\xspace}
\definecolor{todored}{rgb}{0.72, 0.13, 0.13}
\newcommand{\boyu}[1]{\nbc{@Boyu}{#1}{orange}\xspace}

% English
\newcommand{\adhoc}{\hbox{\emph{ad hoc}}\xspace}
\newcommand{\cf}{\hbox{\emph{cf.}}\xspace} % confer which means compare
\newcommand{\deletia}{\ldots [deletia] \ldots}
\newcommand{\etal}{\hbox{\emph{et al.}}\xspace}
\newcommand{\eg}{\hbox{\emph{e.g.}}\xspace}
\newcommand{\ie}{\hbox{\emph{i.e.}}\xspace}
\newcommand{\scil}{\hbox{\emph{sc.}}\xspace} %scilicet: it is permitted to know
%\newcommand{\st}{\hbox{\emph{s.t.}}\xspace} % Clashes with an existing definition
\newcommand{\wrt}{\hbox{\emph{w.r.t.}}\xspace}
\newcommand{\etc}{\hbox{\emph{etc.}}\xspace}
\newcommand{\viz}{\hbox{\emph{viz.}}\xspace} %videlicet: it is permitted to see


\begin{document}


\title{Whose fault is it anyway? \\Safe Integration of LLM-Generated Code}
\author{Peisen Lin}
\affiliation{
	\institution{National University of Singapore}
	\city{Singapore}
	\country{Singapore}
}
\author{Yuntong Zhang}
\affiliation{
	\institution{National University of Singapore}
	\city{Singapore}
	\country{Singapore}
}
\author{Andreea Costea}
\affiliation{
	\institution{National University of Singapore}
	\city{Singapore}
	\country{Singapore}
}
\author{Abhik Roychoudhury}
\affiliation{{National University of Singapore}}

\renewcommand{\shortauthors}{XXX}

\begin{abstract}
The advancements of large language models (LLM) is reshaping the development landscape, making the AI-based assistants increasingly suitable for pair-programming. 
%
Developers write a prompt in natural languages describing the desired functionality and the AI assistant generates the requested code which is then integrated into existing code bases. 
%
Since these prompts generally only describe the functionality of the desired code, the LLM has little information about the non-functional requirements of the code base it is going to be integrated in. 
%
In other words, the auto-generated code contains implicit non-functional assumptions about its later use which may not align with those the developer made, e.g. whether a pointer parameter can be NULL or not. 
%
Obviously, such misalignment in assumptions could lead to issues in the existing code base when the auto-generated code gets integrated in.
%
In case of misalignment, a natural question to ask then is what needs to be repaired? The existing code base or the auto-generated code?
%
Stated differently, we are seeking to assign the blame for possible faults and  to devise possible repair strategies. 
%
We show the role Incorrectness Logic could have in automatically extracting these implicit non-functional assumptions in the auto-generated code and render them explicit to facilitate a safer code integration.
%
In particular, we are looking at the problem of code comprehension not from the traditional functionality angle, but from that of safety properties. 
%
We propose a safe integration framework which uses static analysis (1) to detect possible memory safety issues in the auto-generated code, (2) to describe these possible issues logically, and then (3) to identify which program unit is responsible for fixing the issues in order to generate sanitisers which ensure that the safety requirements in the existing code align with the safe usage of the auto-generated code.
% 
%As a use case showing the utility of such a workflow, we instantiate this framework to ensuring that the integration of auto-generated code with existing code bases does not lead to any Null Pointer Dereference. 
\end{abstract}


\begin{CCSXML}
<ccs2012>
 <concept>
  <concept_id>00000000.0000000.0000000</concept_id>
  <concept_desc>Do Not Use This Code, Generate the Correct Terms for Your Paper</concept_desc>
  <concept_significance>500</concept_significance>
 </concept>
 <concept>
  <concept_id>00000000.00000000.00000000</concept_id>
  <concept_desc>Do Not Use This Code, Generate the Correct Terms for Your Paper</concept_desc>
  <concept_significance>300</concept_significance>
 </concept>
 <concept>
  <concept_id>00000000.00000000.00000000</concept_id>
  <concept_desc>Do Not Use This Code, Generate the Correct Terms for Your Paper</concept_desc>
  <concept_significance>100</concept_significance>
 </concept>
 <concept>
  <concept_id>00000000.00000000.00000000</concept_id>
  <concept_desc>Do Not Use This Code, Generate the Correct Terms for Your Paper</concept_desc>
  <concept_significance>100</concept_significance>
 </concept>
</ccs2012>
\end{CCSXML}

\ccsdesc[500]{Do Not Use This Code~Generate the Correct Terms for Your Paper}
\ccsdesc[300]{Do Not Use This Code~Generate the Correct Terms for Your Paper}
\ccsdesc{Do Not Use This Code~Generate the Correct Terms for Your Paper}
\ccsdesc[100]{Do Not Use This Code~Generate the Correct Terms for Your Paper}

%%
%% Keywords. The author(s) should pick words that accurately describe
%% the work being presented. Separate the keywords with commas.
\keywords{program termination, program repair, program synthesis}


\maketitle

\section{Introduction}
\label{sec:introduction}

%usecase
State of programming these days.(copilot)

Motivating example (copilot)




\section{Overview}
\label{sec:overview}
\begin{itemize}
\item The analysis of the vendor code

\item From summary to higher-order sanitiser

\item Analysis of the client code

\item Instantiation of sanitiser

\end{itemize}

\section{Approach}
\label{sec:approach}
\input{sections/approach.tex}
\section{Evaluation}
\label{sec:evaluation}
Does the vendor code make safety assumption of client code?

Can our approach accurately make the implicit assumptions explicit?

Can our approach avoid unsafe usage of vendor code?

Experiment plans:
Test our approach on library usage

Test our approach on LLM-generated vendor code

Implementation TODOs:
1. Implement the instantiation from higher-order sanitiser to C code (by 27th Feb)
2. Automate the entire tool (minor)
3. For the test of libraries, modify infer to store some external calls in infer database (along with 4)
4. Testing for both usage, library usage takes longer as we need to build the client for infer (ideally by 5th, at least one library tested, or LLM tested) (by 15th)
\section{Related Work}
\label{sec:related_work}
\input{sections/related_work.tex}
\section{Conclusion}



\bibliographystyle{ACM-Reference-Format}
\bibliography{bibliography}

\end{document}
